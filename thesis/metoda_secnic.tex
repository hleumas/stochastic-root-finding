\section{Stochastická metóda regula falsi}
\subsection{Motivácia}
Spomedzi numerických metód hľadania koreňa patrí regula falsi k~najstarším.
Prvé zmienky o~nej sa vyskytujú už v~treťom storočí pred našim letopočtom
v~indickom diele Vaishali Ganit. 

Regula falsi funguje podobne ako metóda polenia intervalov. Štartuje s~dvoma
bodmi $a_0$, $b_0$ takými, že $f(a_0)$ a $f(b_0)$ sú opačných znamienok. Za
predpokladu spojitosti analyzovanej funkcie $f$ je potom v~intervale 
$[a_0,b_0]$ zaručená existencia koreňa. Metóda produkuje postupnosť intervalov
$[a_k, b_k]$ aproximujúcich hodnotu koreňa. V~$k$-tom kroku sa spočíta hodnota

\[c_k = {f(b_k)a_k - f(a_k)b_k\over f(b_k) - f(a_k)} \label{regula:iter}\]

Možno ľahko nahliadnuť, že $c_k$ je priesečníkom $x$-ovej osi a spojnice bodov
$(a_k, f(a_k))$ a $(b_k, f(b_k))$. V~nasledovnom kroku sa jeden z~bodov $a_k$,
$b_k$ nahradí bodom $c_k$ a druhý sa ponechá tak, aby bolo naďalej splnené, že
$a_{k+1}$ a $b_{k+1}$ majú opačné znamienka. Tento postup sa iteruje dokým
nedostaneme dostatočne presnú aproximáciu koreňa.

\subsection{Stochastická verzia}
Celú metódu budeme formulovať pre rastúcu $f$ a binárny experiment s~hodnotami
$\pm 1$. Ak by sme pustili originálnu verziu regula falsi na SRFP, algoritmus by
síce k~nejakej hodnote konvergoval (v~každom kroku by sa interval zmenšil na
polovicu), ale nebolo zaručené (a ani pravdepodobné), že táto hodnota by bola
koreňom skúmanej funkcie. Aby sme si toto poistili, metódu jemne modifikujeme
tak, aby v~každej iterácii merala hodnoty v~daných bodoch viackrát a aby tento
počet stúpal s~každou iteráciou. Takto algoritmus získa vždy dostatočne presné
aproximácie hodnôt funkcie a bude zaručené konvergencia ku koreňu.

Nasleduje podrobný popis fungovania algoritmu:
\begin{itemize}
\item Rastúca funkcia $s(k)$ udáva závislosť počtu meraní od čísla iterácie
$k$

\item V~každej iterácii algoritmus disponuje poľom trojíc $(x, y, c)$, kde $x$ sú
všetky doteraz skúmané body, $y$ sú aproximácie hodnoty $f(x)$ vytvorené z~$c$
meraní. Toto pole trojíc je usporiadané podľa hodnôt $x$. Ďalej si udržiava
informáciu o~práve analyzovanom bode $x_\mrm{cur}$.

\item Algoritmu štartuje s~dvoma hodnotami $x_0, x_1$, pre ktoré je zaručené, že
$f(x_0) < 0$ a $f(x_1) > 0$ a v~poli sú zaradené ako $(x_0, -1, \oo)$ a $(x_1,
1, \oo)$. Práve analyzovaný bod $x_\mrm{cur}$ je inicializovaný na $x_0$.

\item V~$k$-tej iterácii algoritmus odmeria $s(k)$-krát hodnotu v~bode
$x_\mrm{cur}$. Na základe znamienka priemeru usúdi, či sa jedná o~bod vpravo
alebo vľavo od hľadaného koreňa. Ďalej vezme najbližší bod $x$ vľavo resp.
vpravo od $x_\mrm{cur}$. K~nemu príslušná hodnota $y$ má určite opačné znamienko
ako hodnota práve spočítaná hodnota $y_\mrm{cur}$. 

\item V~takto vybratom bode $x$ algoritmus odmeria hodnotu toľkokrát, aby počet
meraní v~tomto bode bol aspoň $s(k)$. V~prípade, že sa znamienko $y$ týmito
meraniami zmení, algoritmus nastaví novú hodnotu $x_\mrm{cur}$ na $x$ a celý
postup zopakuje. 

\item Ak hodnota $y$ ani po dodatočných meraniach nezmení znamienko, je
zaručené, že $y$ a $y_\mrm{cur}$ majú opačné znamienka. Podľa vzťahu
\eqref{regula:iter} vypočítame novú hodnotu $x_\mrm{cur}$ a prejdeme do $(k+1)\.$
iterácie.
\end{itemize}

\subsection{Dôkaz konvergencie}
Ukážeme, že popísaný algoritmus konverguje podľa pravdepodobnosti ku koreňu
hľadanej funkcie $f$. Najskôr však uvedieme niekoľko pomocných tvrdení.

\begin{lema}
Nech $Y_1, Y_2, \dots$ sú nezávislé Bernoulliho náhodné premenné so strednou
hodnotou $p>\tfrac{1}{2}$ a $S_{n} = \tfrac{1}{n}\sum_{i=1}^n Y_i$. Potom pre
ľubovoľné $\eps$ existuje také $k$, že: 

\[P\biggl[\bigcup_{i=k}^\oo(S_{i}\leq\tfrac{1}{2})\biggr] < \eps
\label{regula:chernoff}\]
\end{lema}
\begin{proof}
Podľa Chernoffovej nerovnosti platí
\[P(S_i\leq\tfrac{1}{2}) < \exp[-2n(p-\tfrac{1}{2})^2]\,\.\]
Podľa union bound nerovnosti potom možno \eqref{regula:chernoff} odhadnúť ako
\[\sum_{i=k}^\oo \exp[-2n(p-\tfrac{1}{2})^2]\]
Pre pravdepodobnosť z~\eqref{regula:chernoff} dostávame odhad
\[P < {\exp[-2k(p-\tfrac{1}{2})^2] \over 1 - \exp[-2(p-\tfrac{1}{2})^2]}\,\.\]
Keďže $p-\tfrac{1}{2} > 0$
\[\lim_{k\to\oo}{\exp[-2k(p-\tfrac{1}{2})^2] \over 1 -
\exp[-2(p-\tfrac{1}{2})^2]} = 0\,\.\]
Čím je dôkaz dokonaný.
\end{proof}

\begin{dosledok}
\label{regula:neutec}
Pre každé $\eps>0$ a $d>0$ existuje $k\in\N$, že ak v~$k$-tej alebo neskoršej
iterácii algoritmu platilo
\[\abs{x_\mrm{cur}-\x}<d \label{regula:close}\,,\]
tak pravdepodobnosť, že pre nejakú neskoršiu iteráciu by táto podmienka prestala
platiť, je menšia ako $\eps$.
\end{dosledok}
\begin{veta}
Nech $s(k) \geq k$. Potom v~stochastickej verzii regula falsi konverguje 
$x_\mrm{cur}$ k~$\x$ takmer určite.
\end{veta}
\begin{proof}
Nech $d>0$. Označme $\beta$ menšiu z~hodnôt $\abs{f(\x-d)}$ a $\abs{f(\x+d)}$.
Keďže $f$ je rastúca, tak pre $x$ mimo intervalu $[\x-d, \x+d]$ platí
\[P[F(x) = \sgn f(x)] \geq \tfrac{1}{2}(\beta + 1)\,.\] 
Horeuvedený vzťah popisuje pravdepodobnosť dostať správne znamienko pri jednom
meraní v~bode $x$. 

V~$k$-tej iterácii je každá použitá hodnota aproximovaná priemerom $s(k)>k$
hodnôt.
Podľa Chernoffovej nerovnosti je preto
pravdepodobnosť namerať v~$k$-tej iterácii nesprávne znamienko pre $x$ mimo
intervalu $[\x-d, \x+d]$ menšia ako
\[\exp[-\tfrac{1}{2}k\beta^2]\,\.\label{regula:sign}\]

Hodnôt $x$ mimo intervalu $[\x-d, \x+d]$ je v~$k$-tej iterácii najviac $k$.
Pravdepodobnosť, že by sme v~tejto iterácii pre niektorú z~nich určili nesprávne
znamienko je preto menšia ako
\[k\exp[-\tfrac{1}{2}k\beta^2]\,\.\]
Nepochybne vieme ku každému $\eps>0$ nájsť také $k$, že horeuvedený výraz bude
menší ako $\eps$. To však znamená, že v~$(k+1)\.$ iterácii je
s~pravdepodobnosťou $1-\eps$ bod $x_\mrm{cur}$ od $\x$ vzdialený menej ako $d$. 

Potiaľto sme dokázali konvergenciu podľa pravdepodobnosti k~$\x$. S~využitím
dôsledku \eqref{regula:neutec} však dostávame silnejšiu konvergenciu takmer
isto.
\end{proof}

\subsection{Voľba funkcie $s$}
Predošlý výsledok ukazuje, že algoritmus takmer isto konverguje k~$\x$ pre
ľubovoľnú rozumnú funkciu $s$. Napriek tomu má zmysel zamýšľať sa nad jej
voľbou, pretože tá bude mať veľký vplyv na rýchlosť algoritmu. Pre príliš malé
$s$ sa bude algoritmus často mýliť a bude preto často merať v~nesprávnych
bodoch, pre príliš veľké $s$ však naopak bude merať zbytočne veľa hodnôt
v~každom bode.

Dobrú voľbu $s$ môžeme skúsiť odhadnúť nasledovnou úvahou. V~blízkom okolí
koreňa funkcie bude vzdialenosť od koreňa približne priamo úmerná veľkosti
funkcie v~danom bode. Za optimálnych podmienok môžeme očakávať, že v~priemere sa
každou iteráciou vzdialenosť od koreňa zmenší na polovicu, tým pádom pre veľkosť
funkcie v~$k$-tej iterácii platí približne vzťah $f(x_k) \propto 2^{-k}$.
Podľa Chernoffovej nerovnosti potrebujeme na určenie znamienka v~$k$-tej
iterácii počet meraní úmerný $2^{2k}$. To nasvedčuje, že z~asymptotických
dôvodov, je rozumné zaoberať sa funkciami, ktoré sú exponenciálne od $k$.
V~praxi treba opäť zvoliť istý kompromis medzi asymptotikou a chovaním v~prvých
krokoch, kedy funkciu nemožno aproximovať priamkou. Vtedy, hlavne v~prípadoch
kedy sú na krajoch dlhé oblasti hodnôt blízkych k~1 a -1, je priveľké $s$
plytvaním. Sú tu ďalšie možnosti zlepšovania algoritmu tak, aby si $s$ volil
dynamicky na základe svojich meraní, ale tie idú mimo rozsah tejto práce.
