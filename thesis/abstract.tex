\noindent SRFP (Stochastic Root Finding Problem), špeciálne jeho binárna verzia,
je problém, ktorý sa opakovane zjavuje v~rôznych oblastiach vedeckej činnosti.
Od roku 1951, kedy Robbins a Monro \cite{monro} predstavili prakticky prvý
stochastický algoritmus na jeho riešenie, vyšlo množstvo článkov teoreticky
analyzujúcich tento a iné algoritmy. Napriek tomu je možné nájsť veľmi málo
prác, ktoré by skúmali správanie sa a vhodnosť týchto algoritmov v~reálnom
nasadení. Teoretické výsledky o~asymptotickom správaní sa algoritmov a
konvergencii sú nepochybne istým indikátorom jeho kvality, nedokážu však nikdy
plne zodpovedať otázku, ako sa bude algoritmus správať v~praxi. Preto sme sa
v~tejto práci zamerali na praktické otestovanie algoritmov pri riešení binárneho
SRFP a pomôcť tak pri ich reálnom nasadení.

Merania poukázali na slabiny algoritmov, Stochastická verzia Newtonovho
algoritmu \cite{anbar} sa dokonca ukázala úplne nepoužiteľnou. Po zakomponovaní
istých heuristík sa nám ju však podarilo podstatne vylepšiť. V~práci sme taktiež
predstavili vlastný návrh algoritmu s~veľkým potenciálom ďalších zlepšení, ktorý
sa za niektorých podmienok ukázal ako výkonnostne najlepší spomedzi algoritmov.

\noindent\textbf{Kľúčové slová}: Stochastic root finding problem, Robbins-Monro,
Stochastic Newton-Rapshon
