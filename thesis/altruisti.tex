\section{Motivácia binárneho SRFP}
Algoritmy, ktoré sme si zatiaľ predstavili, boli navrhnuté pre ľubovoľný
experiment s~ohraničenými hodnotami výsledkov. To je nepochybne veľmi užitočná
vlastnosť týchto algoritmov, v~praxi sa však vo veľkej miere stretávame so
situáciami, ktoré majú binárny charakter. Študenti testujú či sa im podarí
obhájiť bakalársku prácu, alebo nepodarí, pyrotechnika zaujíma čí bomba
vybuchne, alebo nevybuchne a lekára trápi, či sa pacient uzdraví, alebo zomrie.
Preto sme sa rozhodli v~tejto práci zamerať na aplikáciu SRFP algoritmov na
problémy s~binárnymi výsledkami. Doteraz sme uviedli len veľmi všeobecné
príklady reálnych aplikácii, na konkrétnom príklade si preto teraz predvedieme
užitočnosť binárneho SRFP pri riešení otázky
\subsection{Prečo ľudia konajú dobro?}
Na riešenie tohto problému sa podujali Tomáš Kulich a Jaroslav
Flegr\cite{kulich}. Ako vo svojej práci uvádzajú, pôvod altruistického
chovania, teda chovania užitočného pre spoločnosť ako celok, avšak nevýhodného
pre jedinca, bol dlho orieškom pre evolučných biológov. Skupine s~altruistickými
jedincami sa ako celku darí lepšie než ostatným skupinám, no altruistický
jedinci v~skupine sú oproti sebcom znevýhodnení a postupne vymierajú.
Pred nejakou dobou vznikla hypotéza, že ak altruizmus kontroluje viacero na sebe
nezávislých génov, ktorých vhodná kombinácia vytvorí altruistického jedinca,
môže v~spoločnosti existovať stabilná rovnováha medzi altruistami a sebcami. 

Autori sa pokúsili overiť toto tvrdenie numerickou simuláciou jednoduchého
modelu veľkej metapopulácie, ktorá sa členila na množstvo
menších populácii. Monitorovali chovanie sa metapopulácie v~priebehu množstva
generácii, kde každá generácia pozostávala z~troch fáz: prirodzený výber,
migrácia a mutácia. Samotný prirodzený výber zachovával počet členov celej
metapopulácie, no zvýhodňoval altruistické populácie pred sebeckými. Počet
potomkov narodených do populácie bol úmerný počtu členov populácie a podielu
altruistov v~nej. V~momente, kedy niektorá populácia klesla v~počte členov pod
dvoch, automaticky ju pohltila najväčšia existujúca populácia.

V~rámci samotných populácii sa však potomok narodil s~väčšou pravdepodobnosťou
sebeckému jedincovi než altruistovi. Veľkosť zvýhodnenia sebcov v~rámci skupiny
a veľkosť zvýhodnenia altruistických skupín pred ostatnými skupinami bola
popísaná parametrami $(\alpha, \beta)$ a v~práci bolo skúmané, pre akú najmenšiu
hodnotu parametra $\alpha$ je aspoň polovičná pravdepodobnosť nájsť po $N$
generáciách v~metapopulácii aspoň nejaké malé množstvo $\eps$ altruistických
jedincov.

V~tomto prípade sa jednalo o~binárny problém závislý od parametra $\alpha$,
funkcia pravdepodobnosti $f(\alpha)$ nebola známa a jej jednotlivé merania boli
časovo veľmi náročné. Nameranie pár stoviek hodnôt zabralo týždeň počítania
tridsaťjadrového clustera.

Podobné problémy, pri ktorých treba riešiť binárne SRFP, sa objavujú v~rôznych
oblastiach vedeckej činnosti a je preto oprávnené zaoberať sa výkonnosťou SRFP
algoritmov pri ich riešení. 

\subsection{Definícia binárneho SRFP}
V~ďalšom budeme za binárne SRFP považovať SRFP pri ktorom $F$ môže nadobúdať len
hodnoty $\pm 1$. Od skúmaných funkcií $f$ preto budeme prirodzene považovať
splnenie podmienky $\abs{f(x)} < 1$ pre všetky $x$. Takto definované binárne
SRFP nestráca na všeobecnosti, pretože ako čitateľ ľahko nahliadne,
transformáciou $aF + b$ možno tento problém previesť na ľubovoľné binárne SRFP.
