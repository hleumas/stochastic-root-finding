\section{Záver}
V~úvode práce sme si predstavili SRF problém a zoznámili sa s~algoritmami na
jeho riešenie i praktickú motiváciu. Uviedli sme dôkaz konvergencie pre vôbec
prvý SRFP algoritmus od Robbinsa a Monroa\cite{monro} a zoznámili sa so
zaujímavým pokusom o~jeho vylepšenie od Dan Anbara\cite{anbar}. V~tretej
kapitole sme predstavili vlastnú stochastickú adaptáciu na metódu regula falsi a
ukázali, že za podmienok binárneho experimentu rieši SRFP takmer isto. 

V~záverečnej kapitole sme všetky tri algoritmy podrobili dôkladnému
testovaniu, kedy sme preskúmali ich asymptotické vlastnosti, rýchlosť
počiatočnej konvergencie a ďalšie parametre.  Ukázalo sa, že Stochastická
adaptácia Newtonovej metódy\cite{anbar} od Dana Anbara je pre účely binárneho
experimentu v~pôvodnom tvare nepoužiteľná, avšak po drobnej modifikácii sa stala
veľmi výkonnou pri niektorých inštanciách problému. Takto modifikovaná Newtonova
metóda aj metóda regula falsi mali veľmi podobné charakteristiky, v~meraniach sa
však ukazovalo, že regula falsi je vo väčšine prípadov jemne výkonnejšia. Tieto
dve metódy vhodne dopĺňala metóda Robbins-Monro\cite{monro}, ktorá bola
jednoznačne najvýkonnejšou metódou pokiaľ sa jednalo o~konvergenciu na úseku
okolo koreňa, kde bolo možné funkciu dobre lineárne aproximovať. Zvyšným dvom
metódam sa skôr darilo v~prípade prudko meniacich sa funkcií, na ktoré sa vedeli
lepšie adaptovať.
