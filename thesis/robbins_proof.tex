%Lema 1{{{
\begin{lema}
Nech $f$ je neklesajúca náhodná premenná ohraničená konštantou
$C>0$, teda
\[F^2(x) < C^2 \label{rob:ohranicenie}\,\.\]
Potom strednú kvadratickú chybu $n$-tého kroku
\[b_n = \av{(x_n-\x)^2}\]
možno vyjadriť v~tvare
\[b_{n} = b_0 + \sum_{i=0}^{n-1} a_i^2 e_i - \sum_{i=0}^{n-1} 2 a_n
d_i\,,\label{becko}\]
kde 
\begin{align}
e_n &= \av{f^2(x_n) + S^2(x_n)}\,,\\
d_n &= \av{(x_n-\x) f(x_n)}\,\.
\end{align}
\end{lema}

\begin{proof}
Rekurenciu možno prepísať do tvaru\cite{pasupathy}
\[x_{n+1} = x_{n} - a_n f(x_n) - a_n(S(x_n))\,,\label{rekurencia}\]
kde náhodná premenná 
\[S(x_n)=F(x_n)-f(x_n)\,\.\]
Označme
\[b_n=\av{(x_n-\x)^2}\]
Potom možno $b_{n+1}$ vyjadriť pomocou $b_n$.
\[b_{n+1} = \av{\av{(x_{n+1}-\x)^2\mid x_n}}\]
Využitím rekurencie \eqref{rekurencia} výraz zjednodušíme na
\[b_{n+1} = b_n - 2 a_n \av{(x_n-\x) f(x_n)} + a_n^2 \av{f^2(x_n)+S^2(x_n)}\,,\]
kde sme využili fakt, že $\av{S(x_n)\mid x_n}=0$.
Zopakovaním postupu $n$-krát dostaneme požadované vyjadrenie \eqref{becko}.
\end{proof}
%}}}

%Lema 2{{{
\begin{lema}
Limita 
$\lim_{n\to\oo}b_n$
existuje a suma $\sum_{i=0}^\oo a_n d_i$
konverguje.
\end{lema}
\begin{proof}
Z~$\eqref{rob:ohranicenie}$ dostávame podmienku
\[e_n < 2C^2\,\.\]
Z~podmienky \eqref{stvorecsum} potom vyplýva konvergencia sumy
\[\sum_{i=0}^\oo a_i^2 e_i\,\.\]
Uvážením $b_n\geq0$ a využitím vzťahu \eqref{becko} dostávame nasledovný odhad
pre sumu
\[\sum_{i=0}^n a_n d_i \leq \frac{1}{2} \Bigl(b_0 + \sum_{i=0}^\oo a_i^2
e_i\Bigr)\,\.\]
Funkcia $f$ je neklesajúca a $f(\x) = 0$, členy postupnosti $d_n$ musia byť
preto nezáporné a suma $\sum_{i=0}^\oo a_n d_i$ musí konvergovať. Z~vyjadrenia
\eqref{becko} pre $b_n$ potom plynie existencia limity postupnosti $b_n$.
\end{proof}
%}}}

Všimnime si, že $b_n$ a $d_n$ udávajú v~istom zmysle presnosť aproximácie $\x$
pomocou $x_n$. To naznačuje, že veľkosť $d_n$ by mohla súvisieť s~veľkosťou
$b_n$. Predpokladajme preto, že existuje nezáporná postupnosť $k_n$ dostatočne
veľkých, aby
\[\sum_{i=0}^\oo a_nk_n = \oo\,,\label{rob:kdef}\]
ktoré však odhadujú $d_n$ pomocou $b_n$ vzťahom
\[d_n \geq k_n b_n\,\.\label{rob:kodhad}\]
Nasledovná lema ukazuje, že takýto odhad nám umožní dokázať konvergenciu
$x_n\pto\x$.

%Lema 3 {{{
\begin{lema}
Nech postupnosť $k_n$ spĺňa podmienky \eqref{rob:kdef} a \eqref{rob:kodhad}.
Potom
$\lim_{n\to\oo}b_n = 0$
a
$x_n\pto\x$.
\end{lema}

\begin{proof}
Z~podmienky \eqref{rob:kodhad} vyplýva konvergencia sumy
\[\sum_{n=0}^\oo a_nk_nb_n\,,\label{rob:bkonv}\]
no podľa podmienky \eqref{rob:kdef} samotná suma
\begin{equation*}
\sum_{n=0}^\oo a_nk_n
\end{equation*}
nekonverguje. To znamená, že pre dostatočne veľké $n$, musí byť väčšina $b_n$
dostatočne malých, aby dokázali zabezpečiť konvergenciu \eqref{rob:bkonv}.
Formálne
\[\A\epsilon>0,n>0\E k>n:b_k < \epsilon\,.\]
Nula je hromadným bodom konvergentnej postupnosti $b_n$ a preto musí byť
tiež jej limitou.
Ukázali sme, že
\[\lim_{n\to\oo} \av{(x_n-\x)^2} = 0\]
čo je však postačujúca podmienka pre konvergenciu $x_n$ podľa pravdepodobnosti.
\end{proof}

%}}}
Ostáva nám ukázať, že je možné zvoliť vyhovujúcu postupnosť $k_n$. Predtým si
však dokážeme jedno pomocné tvrdenie.

%Lema 4{{{
\begin{lema}
Pre ľubovoľnú kladnú postupnosť $a_n$ je podmienka \eqref{linearsum}
ekvivalentné podmienke
\[\sum_{n=1}^\oo \frac{a_n}{a_1+a_2+\dots+a_{n-1}}=\oo \label{rob:komsum}\]
\end{lema}
\begin{proof}
Prvá z~implikácií je triviálna, ukážeme preto iba
\eqref{linearsum}$\then$\eqref{rob:komsum}. Predpokladajme, že 
\[\sum_{n=1}^\oo
\frac{a_n}{a_1+a_2+\dots+a_{n-1}}=H<\oo\,,\label{rob:komsumpred}\] 
potom existuje $m$, že
\[\sum_{n=1}^m \frac{a_n}{a_1+a_2+\dots+a_{n-1}}>H-\frac{1}{2}\,,\] 
avšak z~\eqref{linearsum} vieme, že existuje $k$
\[2\sum_{n=0}^m < \sum_{n=1}^k\,\.\]
Potom však
\[\sum_{n=m+1}^k \frac{a_n}{a_1+a_2+\dots+a_{n-1}} >
\frac{\sum_{n=m+1}^ka_n}{\sum_{n=0}^m a_n} > \frac{1}{2}\]
To však znamená
\[\sum_{n=1}^k \frac{a_n}{a_1+a_2+\dots+a_{n-1}}>H\,,\] 
čo je spor s~predpokladom \eqref{rob:komsumpred}.
\end{proof}
%}}}

\begin{veta}\label{Robbins}
Nech $f$ je neklesajúca funkcia spĺňajúca v~nejakom $\delta$ okolí bodu $\x$
nerovnosť
\[M(x-\x)f(x) > (x-\x)^2 \label{rob:ineq}\]
pre nejaké $M>0$.
Nech zároveň $F$ spĺňa podmienku \eqref{rob:ohranicenie}.
Potom postupnosť definovaná rekurenciou \eqref{rekurencia1} konverguje podľa
pravdepodobnosti k~$\x$.
\end{veta}

\begin{proof}
Pre $\lvert x_n - \x\rvert < \delta$ z~\eqref{rob:ineq} platí odhad
\[\lvert x_n - \x\rvert\rvert < Mf(x_n)\]

Ak je $\lvert x_n - \x \rvert > \delta$, dostávame z~triviálnej nerovnosti
\[\lvert x_n - \x\rvert < \lvert x_0-\x\rvert + C \sum_{i=0}^{n-1} a_n\]
odhad
\[\lvert x_n - \x\rvert < A~f(x_n)\sum_{i=0}^{n-1}a_n\,,\]
kde
\[A =M/a_0 + (C + (\lvert x_0-\x\rvert)/a_0) M/\delta\,.\]
Prvý člen zároveň zabezpečuje platnosť odhadu aj pre prípad
$\lvert x_n - \x\rvert < \delta$.

Tým sme ukázali platnosť nerovnosti
\[(x_n-\x)^2 <  f(x_n)(x_n-\x)A\sum_{i=0}^{n-1}a_n\]
a našli kandidáta pre $k_n$
\[k_n = \Bigl(A\sum_{i=0}^{n-1}a_n\Bigr)^{-1}\]

Stačí ukázať, že podmienka \eqref{rob:kdef} je splnená, teda
\[\sum_{n=1}^\oo \frac{a_n}{a_0+a_1+\dots+a_{n-1}} = \oo \]
to však na základe dokázanej lemy vyplýva z~\eqref{linearsum}.
\end{proof}
\begin{dosledok}
Nech $f$ je neklesajúca funkcia, ktorá má v~bode $\x$ nenulovú
deriváciu a nech $F$ spĺňa podmienku \eqref{rob:ohranicenie}. Potom postupnosť
$x_n$ definovaná rekurenciou 
\[x_{n+1} = x_{n} + F(x_{n})k/n\,,\label{rob:commonform}\]
kde $x_0$ a $k$ sú pevne zvolené reálne čísla, konverguje podľa
pravdepodobnosti k~$\x$.
\end{dosledok}
\begin{proof}
Rekurencia z~tvrdenia je zodpovedá rekurencii \eqref{rekurencia1} s~$a_n=k/n$.
Je jednoduché overiť, že takto definované $a_n$ vyhovuje podmienkam
\eqref{linearsum} aj \eqref{stvorecsum}. Stačí teda overiť vlastnosti $f$, tá má
v~bode $\x$ deriváciu $f'(\x) > 0$. Existuje teda $\delta$ okolie
$\x$ na ktorom $f$ spĺňa \eqref{rob:ineq} s~konštantou $M=2/f'(\x)$ a môžeme
uplatniť vetu \ref{Robbins}.
\end{proof}
